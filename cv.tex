% CV
\documentclass[12pt]{article}

\usepackage[top=0.5in, bottom=0.5in, left=0.5in, right=0.5in]{geometry}
\usepackage{enumitem}
\usepackage{titlesec}
\usepackage{hyperref}
\hypersetup{
  colorlinks=true,
  linkcolor=blue,
  filecolor=magenta,
  urlcolor=blue
}

\titleformat{\section}
{\huge \bfseries \lowercase}{}{0em}{}[\titlerule]
\titleformat{\subsection}
{\Large \bfseries \lowercase}{}{0em}{}[\titlerule]

%==================================================================================
\begin{document}
\begin{center}
\thispagestyle{empty}
\textbf{\Huge Vikas \\}
\normalsize{\small vikaskaushikkarora@gmail.com}\\
\normalsize{\small +918708928110}\\
\vspace{0.5em}
\end{center}
%----------------------------------------------------------------------------------
\section*{EDUCATION}
\textbf{Doctor of Philosophy (Physics)} \hfill \textbf{(2023-Present)}\\
Statistical Physics of Biological Systems with application to Bacterial Physiology\\
Indian Institute of Technology, Delhi \\ \\
\textbf{Masters in Physics} \hfill \textbf{(2020-2022)}\\
Panjab University \\
Sector 14, Chandigarh \hfill  CGPA : 7.5/10 \\ \\
\textbf{Becholars in Physics (Honours)} \hfill \textbf{(2017-2020)}\\
Panjab University, Sector 14, Chandigarh \\
Major : Physics \hfill  CGPA : 8.0/10 \\
Minors : Mathematics, Chemistry \\ \\
\textbf{Class XII} \hfill \textbf{(2017)}\\
Board of School Education Haryana\\
Subjects : English, Chemistry, Mathematics, Physics, Sanskrit \hfill 96.8 \% \\ \\
\textbf{Class X} \hfill \textbf{(2015)}\\
Board of School Education Haryana\\
Subjects : English, Hindi, Mathematics, Sanskrit, Science, Social Science \hfill 98.6 \% \\

%----------------------------------------------------------------------------------
\section*{COMPETETIVE EXAM DETAILS}
\begin{itemize}
        \item{\bfseries Qualified GATE Physics 2023 Exam with AIR 45 and GATE Score of 761 out of 1000}
        \item{\bfseries Qualified NET-JRF Physics June 2022 Exam with AIR 58 and marks 123.125 / 200}
        \item {\bfseries Qualified JEST Physics 2023 Exam with AIR 86 and percentile 98.34}
\end{itemize}

%----------------------------------------------------------------------------------
\section*{AWARDS AND HONOURS}
\begin{itemize}
    \item{\bfseries Among 1 percent national level toppers of  IAPT-NGPE Exam 2019}
	\item{\bfseries State Board Topper for XII class, 8th Rank}
	\item{\bfseries State Board Topper for X class, 3rd Rank}
\end{itemize}

%---------------------------------------------------------------------------------
\section*{SKILLS}
\begin{tabular}{l@{\hskip 0.5in}l}
    Programming Languages & \textbf{Python \& C and some others}\\
	\vspace{0.5em}
    Markup Languages & \textbf{LaTeX \& HTML}\\
	\vspace{0.5em}
    \textbf{Simulations} & made some Physics and Maths simulations using Python \href{https://drive.google.com/drive/folders/111GkPUsfI-T3w1g3KbAKO7RY5q6JeLaL?usp=sharing}{(Link)}\\
	\vspace{0.5em}
    \textbf{Neural Networking} & familiar with basics of Machine Learning and Neural Networking \href{https://github.com/vikaskaushikkarora/neuralNetworks}{(Link)}\\
	\vspace{0.5em}
    \textbf{Linux} & familiar with Linux Operating System\\
\end{tabular}\\

%-----------------------------------------------------------------------------------

\section{EXTRA-CURRICULAR}
\textbf{Made a basic working model of `Gas Chamber : Particle Detectors'} \hfill (October 2018)\\
6th IAPT National Student Symposium on Physics\\
Won 2nd prize for best Poster for the same title\\ \\
\textbf{Made a working model of Rain Alarm} \hfill (September 2019)\\
Electronics Project : Rain Alarm\\
(Under the suprevision of Dr. Neeru Chaudhary, Panjab University)\\ \\
\textbf{Poster presented for the title `Various Theories on Gravitation'} \hfill (March 2019)\\
CHASCON Poster Presentation\\
(Organised by Panjab University, Chandigarh)\\ \\
\textbf{Advanced Course on Special Theory of Relativity} \hfill (January 2020 - May 2020)\\
Massive Open Online Course, IIT Kanpur by \textbf{H.C. Verma}\\ \\
\textbf{Basics of Quantum Mechanics} \hfill (August 2019 - November 2019)\\
Massive Open Online Course, IIT Kanpur by\textbf{H.C. Verma}\\ \\
\textbf{Basics of Special Theory of Relativity} \hfill (December 2018 - March 2019) \\
Massive Open Online Course, IIT Kanpur by \textbf{H.C. Verma}\\

%----------------------------------------------------------------------------------
\section*{MASTERS PROJECT WORK}
\textbf{Quantum Information Processing} \hfill (August 2021 - March 2022)\\
\textbf{`` Role of Quantum Computing in Cryptography "} \\
(Under the Supervision of Dr. Prasantha K. Panigrahi, IISER Kolkata)\\ \\
I learnt about various
Classical Cryptographic Schemes, the potential challenge to present cryptographic
schemes by the advent of Quantum Computing and Quantum Key Distribution Schemes
like BB84 Protocall and their modified versions.

%===================================================================================
\end{document}
