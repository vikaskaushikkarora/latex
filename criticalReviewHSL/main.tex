\documentclass[12pt]{article}
\title{\Huge{\textbf{\textcolor{blue}{Critical Review}}}}
\author{\textbf{by Vikas (2023PHZ8310)}}
\date{}

\usepackage{extsizes}  
\usepackage{xcolor}
\usepackage{titlesec} 
\usepackage{graphicx} 
\usepackage[margin=0.5in,top=0.5in]{geometry}
\usepackage{mathptmx}
\usepackage[T1]{fontenc}

\renewcommand{\baselinestretch}{1.5}
\titleformat{\section}{\huge\bfseries}{\thesection}{0.5em}{}[\titlerule]

%=================================================================================
\begin{document}
\maketitle
	
	This is a critical review on the paper entitled \textbf{Fundamental principles in bacterial physiology - history, recent progress, and the future with focus on cell size control (2018)}\cite{p}  by Suckjoon Jun, Fangwei Si, Rami Pugatch and Mattthew Scott. This review paper is written by the frontiers of the field themselves e.g. Suckjoon Jun and Rami Pugatch. The review paper presents an overview of the models in Bacterial Physiology, approaches developed since the early 20th century, the history of the field itself and seeks the answers to the most important questions.\\
	
	The paper tires to answer the questions like : what are the fundamental principles underlying the physiology of bacteria and what are the important parameters (because there is a huge number of parameters) which have been discovered over time in order to find the best coarse-grained models for the bacterial physiology. Of course, such questions are very relevant and are very important for scientists to have a good understanding of bacterial life and how to control their growth in case of a pandemic etc.\\
	
	The review paper is divided into two parts : the first one reviews the golden era of bacterial physiology which is from 1940s to early 1970s and provides a major list of milestones achieved in the field and the second one discusses more recent rediscoveries (after 2000s) of general quantitative principles and further developments in bacterial physiology.\\
	
	The authors talk about three things in the intro. The first is the fact that \textbf{the bacterial growth is intrinsically different from the aging process of human beings}. This was the most important thing to acknowledge by modern science. The cells of Eschericia coli can be maintained in the process of exponential growth for infinite time such that their growth could be understood quantitatively. This thought was discovered by Monad and then refined by Schaechter and Maaloe. The second thing is the major quest in microbial physiology which is to \textbf{understand the fundamental principles underlying biosynthesis} in a given growth environment. The third thing they talk about is the problem of \textbf{choosing proper state variables}. Since there are probably hundreds of thousands of proteins for various purposes inside a bacteria, we can not keep track of each one individually. So we have to decide which ones are the most important and which ones are quite alike such that they can be grouped together in our coarse-grained model.\\
	
	In the next section, the authors talk about the the first golden era of bacterial physiology which is the period between late 1940s and early 1970s. A typical progression during this period was first a new technology allowed novel experiments that were not possible before, followed by modeling efforts to explain the data. The part of this section describes all the technological achievements which had made the groundbreaking discoveries in bacterial physiology possible. There is a mention of \textbf{Carlsberg Pipette} for accurate measurements of volumes upto ten microlitres, \textbf{Chemostat} for serial dilution and maintaining steady state for bacterial culture, \textbf{Radioactive Pulse-Labelling and Autoradiography} for understanding the role of DNA in reproducing new cells, \textbf{Advanced Microscopy} for measuring the cell length and age distribution, \textbf{Baby Machine} to understand multifork-replication of DNA and \textbf{Technological Advancements of Computers} for high level computation and simulation which was essentially required.\\
	
	The authors also discuss the important models and conceptual advancements in the field. They argue that the first conceptual advancement is due to Monod's work. The early work in microbiology was complicated by a lack of well-defined state variables and standard reference conditions. Jacques Monod’s review in 1949 \cite{monod} made a clear case that, \textbf{with the properly-chosen state variables, simple quantitative relations could be derived for steady state growth of bacteria}. Then the authors argue that the most important milestone was achieved by the work of Schaechter and Maaloe in 1958 \cite{sm1,sm2}. They discovered the relationship of bacterial DNA, RNA and mass growth with the growth rate and discovered the first bacterial growth laws owing to their extreme care while taking measurements. The authors compare this work with initial developments of quantum mechanics and call this as \textbf{`The Copenhagen school of bacterial physiology'}. They also mention the Helmstetter and Cooper Model \cite{hc} for bacterial growth. Thier work includes multifork-replication in DNA using baby machine and radioactive pulse labelling recognising the very important work of DNA in cell-reproduction for the first time. Their observations of DNA replication using multifork replication laid the groundwork for the future research in bacterial physiology.\\
	
	The case of cell-age distribution of bacteria in steady-state is also discussed here. They specifically do not mention any papers; they rather say that the work is done by so many people starting from the early days of the famous mathematician Euler such that it is not viable to include so many citations. But I think they should have mentioned at least the most important ones. Neverthless, they provide a really good insight of the importance of relatively simple, but beautiful mathematics in the biology.\\
	
	In the third and fourth section of the paper, the authors talk about the cell size homeostasis i.e. how the cell is able to maintain it's shape while the various biological processes e.g. cell-division, alter it's shape, size and volume. The three models are discussed : sizer, timer and adder. The theoretical predictions of sizer, timer and adder models are discussed first and then they are compared with the experimental data. It is found that for the predictions of adder hold true for real experimental tests and hence this model is the true for real world bacterial growth, whereas the other models are not so viable, but mathematically elegant though.\\

	In the later part, the authors talk about recent discoveries and developments in this field. After 2000s, the \textbf{need of proteome patitioning} was realised and this work is done by Matthew Scott and Terence Hwa et al \cite{2010,2014}. The bacterial growth laws are derived from proteome partitioning and considering the role of limiting amino acids. They provide a better coarse grained model. The better parameters are used as growth rate and ribosomal mass fraction for growth dependent and growth independent proteins. They talk about introducing the control and try to derive Monod Kinetics from their model.\\
    
    This paper is written well and cohesively and it broadly covers all the important discoveries in the field of Bacterial Physiology : the breakthroughs and the recent ones also. The paper raises some important questions in the end which are the basis for the future research. To start with the first one, we already know the bacterial growth laws and how do they emerge from more basic school of thoughts (models), but we don't know \textbf{how do physiological controls evolute with time}. That is, we don't know \textbf{``how bacteria dynamically allocate their resources between different processes while the external conditions change''} because bacteria do live in colonies and just talking about their isolated individual behaviour is not sufficient in real world because other bacterial bodies effect the physiological properties of a single bacterium. Also, the amount of nutrients in the medium might change unlike the controlled steady state growth in labs and that is poorly understood yet. Hence, a big challenge in the future, will be to \textbf{develop `control laws’ just like `growth laws’ that will generalize our understanding of steady-state behavior to a dynamic regime, not just steady state}.\\

    Secondly, the authors argue that since the kinetics of the biosynthesis stochastic, the biomolecular abundance and the physiological parameters will vary from one cell to another. Despite this variablity, every daughter cell still must inherit the same proteins, the similar chromosome, and the similar cell envelope or otherwise it will die. It is unclear how bacteria like E. coli coordinates replication initiation and cell division, and more generally all biosynthesis, to avoid this catastrophe.\\

    Another very interesting idea the authors of this paper provide for future research is the use of thermodynamics to understand the bacterial growth. Intriguingly there seems to be a close relation between thermodynamic yield, growth rate and the manner in which the cell schedules its reproduction. Measuring thermodynamic efficiency of growth can lead to a better understanding of a variety of bacteria. This is a really interesting aspect of bacterial physiology and has never been researched before. \\

\bibliography{sources}{}
\bibliographystyle{plain}
\end{document}

