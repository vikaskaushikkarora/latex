\documentclass[14pt]{article}
\title{\Huge{\textbf{\textcolor{red}{Nolan Versus Tarantino}}}}
\author{\textbf{By Vikas Kaushik}}
\date{}

\usepackage{extsizes} %Use this package for global fontsizes 8,10,11,12,14,17,20 
\usepackage{xcolor} %Use this package for various types of colors
\usepackage{titlesec} %Use this package for changing section , subsection etc formatting and spacing
\usepackage{graphicx} %Use this package to include images
\usepackage[margin=0.8in,top=0.8in]{geometry} %Use this package to set margins

\titleformat{\section} %Change the format of Section Headings 
{\huge\bfseries}
{\thesection} %You can also write '$/bullet$' in here !'thesection' is for numbering before headings
{0.5em}
{}[\titlerule]
\titlespacing{\section} %Change the spacing left , up or down to the section heading
{0in}{0.5in}{0.4in}
\titleformat{\subsection}
{\color{blue}\bfseries\Large}{}{0em}{}
\titleformat{\subsubsection}
{\color{red}\bfseries\large}{}{0em}{}


%====================================================
\begin{document}
\maketitle



%||||||||||||||||||||||||||||||||||||||||||||||||||||(Introduction)-1
\section{Introduction}
	\begin{figure}[h]
		\centering
		\begin{minipage}[b]{0.4\textwidth}
			\includegraphics[width=\textwidth,height=0.4\textheight]{reqd/nolan.jpg}
			\caption{Christopher Nolan\label{nolan1}}
		\end{minipage}
		\hfill
		\begin{minipage}[b]{0.4\textwidth}
			\includegraphics[width=\textwidth,height=0.4\textheight]{reqd/tarantino.jpg}
			\caption{Quentin Tarantino\label{tarantino1}}
		\end{minipage}
	\end{figure}
	Every experienced director is a good director in himself/herself and I appreciate many of them :- Martin Scorsese , Stanley Kubrick , Satyajeet Ray , Christopher Noaln , Quentin Tarantino , Joss Whedon , Zack Snyder , James Cameron , James Wan , Russo Brothers , Anurag Kashyap , Priyadrashan and many others , I will run out of pages if I start to name them ! But two of them with extraordinary talent to direct movies in a specific way that I like are Christopher Noaln and Quentin Tarantino . They are both very good directors , possess a lot of experience in movie making and without a doubt , very talented . But there are very subtle differences in the direction style of both of them which we are going to discuss here .



%||||||||||||||||||||||||||||||||||||||||||||||||||||(Cinematography)-2	
\section{Cinematography}
%====================================================(Quentin Tarantino)
	\subsection{Quentin Tarantino}
		\begin{figure}[h]
			\centering
			\includegraphics[width=\textwidth]{reqd/bradpitt.jpg}
			\caption{A Scene from Once Upon A Time In Hollywood \label{bradpitt1}}
		\end{figure}
		\subsubsection{Steady Camera}
			When we are talking about the cinematography , a Quentin Tarantino movie is an absolute winner . The reason is quite simple : `` Quentin just knows how to handle the camera very well . '' The very first thing in his movies is a still camera . `` Quentin just doesn't want the camera to be moved , even a little bit . '' He just places the camera at some place and records naturally , no movements of the viewing angle after that which seems so natural . Even the jump from one camera to another is very little . ( A lot of directors fail at that , Tarantino is an old man in this game since 1992 !) The very best thing about this feature of steady camera with a very little swtiching from one view to the other is that the viewer can easily navigate through the space and time inside the movie , for example I still have directional sense of space for the places inside the movies `The Reservoir Dogs' and `The Hateful Eight' whereas I have such serious problems to do that in Nolan Movies ! The steady camera and and a little number of different viewing angles just makes the place look realistic and easy to remember .
		\subsubsection{Continuous Shots}
			Quentin also takes good continuous shots . I mean , he is master of that , e.g. The Pulp Fiction , The Hateful Eight , Once Upon a time in Hollywood and The Inglorious Bastards . I admire Quentin Tarantino for that . I am pretty sure that the cinematography of the movie `1917' would have given Tarantino a boner because the whole movie is a continuous shot ! 
		\subsubsection{Variety of Shots}
			In spite of continuous shots , Quentin efficiently takes close-up shots , wide angle shots , far-away shots . He knows exactly when to zoom in the camera to the actor's face and capture his/her emotions ; he knows exactly when to zoom out and make the whole shot look like plot-centric . This is a very good quality and if you don't know that , the whole movie becomes mess !
%=====================================================(Christopher Nolan)
	\subsection{Christopher Nolan}
		Christopher Nolan is not known for the cinematogrphy , let's be honest ! The cinematography is very bad in Tenet , let's face it . But this doesn't mean that he doesn't know about a good cinemtography . The cinematography in Dunkirk is awesome ! It really captures the emotions of the soldiers stuck at the island and their struggle .But it just so happens that Nolan cares about other stuff in his movies ! He uses the cinematography just to tell the story . The main reason for ` not the best cinematography ' in Nolan movies is that the story of almost every Nolan movie is very long and the plot is always so complex . Nolan just can not give actors so much screen time with close-up shots  and he can not just make the camera steady for a good cinematography because otherwise the movie will be 10 hours long ! 



%||||||||||||||||||||||||||||||||||||||||||||||||||||||(Script, Plot and Story)-3
\section{Scrpit , Plot and Story}
%=====================================================(Christopher Nolan)
	\subsection{Christopher Nolan}
		Christopher Nolan is known for his script writting ; his script for any movie is way good . The story of his movies are very complicated and often messes up our daily life intuition of space and time . The way of story-telling by Nolan is very unique and non-linear , which can be seen extensively in Memento and Tenet . The story of his movies , as told earlier are quite complicated and very hard to understand in the first go . The stories of his movies are either science-fiction or fiction related . However , Insomnia and Dunkirk are some exceptions . Christopher Nolan takes care of minute details in his movies . The Batman Begins provides a good example , which portrays the origin of a superhero billionare vigilante very well and realistic details . Details and accuracy are an important part of this director's job ! He never misses them . Consider the science-fiction movie Interstellar directed by Christopher Nolan ; the work done by the Nobel Prize winning physicist Kip Thorne ( just to make a simulation of a blackhole so that they can add the futtage of that in the movie) has proven to be the base for some real physics research papers in blackhole imaging ! Now , that is the level of details and accuracy we are talking about .
		\\\\Now , consider another scene of the aeroplane crash in the movie Tenet . That aeroplane crash is 100 percent real ! Nolan had the option to use VFX and computer graphics in order to make the scene look like an aeroplane crash , but this director hates VFX so much that he prefers a real crash ! Although he argued that the VFX budget for the scene would have cost the producer more money than a real crash , but still other good directors like Zack Snyder and Russo Brothers would have preferred VFXs instead of carrying the burden of a real crash !
		\\\\This is not the only case of this kind . Earlier during the shooting of a scene in the movie ` The Dark Knight Rises ' where some bunch of trained people cut down the aeroplane in the air ! And that scene is also real withuot using any VFX ! I wonder why Nolan hates VFX so much ; it would have saved a bunch of aerplanes if Nolan had loved VFX !
		\\\\Deatils are important to Nolan , wether they are visual details or in the conversation of the characters in the movie . Nolan takes care of the script everytime . If Tarantino gives actors more screen-time for character development , the script does the same for Nolan . Almost no scenes in the Nolan movie seem logically incorrect . Why ? Becuase he has got a good script . Consider The Dark Knight . The script is just awesome ! The Batman just does not appear because there should be some fight in the movie ! It is because he is needed by the other characters like Commisnor Gordon and Harvey Dent to do the things that only Batman can do . And that's where we know that the converstation between the characters of the movie has really paid off to add details in the story and proceed it . The Prestige is also a good example of a great script .
%=======================================================(Quentin Tarantino)
	\subsection{Quentin Tarantino}
		The plot of a Quentin Tarantino movie is unique and realsitic in the begining , but may provide us with the view of an alternate parallel universe in which Hitler is murdered inside a theatre and Sharon Tate lives ! The plot of a Quentin Tarantino movie is diverse , with a lot of characters in it whose timelines intersect in the future . e.g. Pulp Ficton , Once Upon A Time In Hollywood , Inglorious Bastards , Kill Bill etc. Tarantino also adds details , where camera also does the work for that , like showing the feet of people (This guy is in love with a good feet !), the kind of clothes they wear , characters eating food , characters cursing each other . However , the conversation also plays an equal good role to proceed the story e.g. the scene where Hans Landa convinces the poor german farmer to hand him over the french family ( I mean this scene has a Tarantino mark on itself !) . 
		\\\\Tarantino is completely able to show a very good drama , the script is way good and imaginary too . Quentin Tarantino also accepts that he is also a fan of a great script . Story-telling of a Tarantino movie is completely drama type , this style is inbuilt in himslef ! You know , when you are walking on a road or sitting inside your car and observing the world around you , it's a Quentin Tarantino movie in itself ! His stroy-telling is too natural , although non-linear sometimes . ( Unlike Christopher Nolan who makes too much intellectual movies ! ) 



%|||||||||||||||||||||||||||||||||||||||||||||||||||||||(Realism and Ficton)-4
\section{Realism and Fiction}

	
\end{document}
